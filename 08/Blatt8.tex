\documentclass{article}
\usepackage{graphicx} % Required for inserting images
\usepackage{amsmath}
\usepackage{amssymb} % used for math symbols
\usepackage{mathtools}
\usepackage{enumitem}
\usepackage{wasysym}
\usepackage{float}
\usepackage{minted}


\title{Computer Grafik Blatt 8}
\date{June 2023}

\begin{document}

\maketitle

\section*{Aufgabe 1.}

\subsection*{(a)}

Gesucht: In wie vielen Faces liegt im Durchschnitt ein Vertex bei einem Hexagon-Netz? \\

Anzahl der Halbkanten: $H = 2 \cdot E$ \\
$6F = H \Rightarrow 6F = 2E \Rightarrow 3F = E$ \\

Mit Euler Formel nach E umstellen: \\
\[
    \begin{aligned}
        V - E + F &= 2 \cdot (1 - g) \\
        6V - 6E + 6F &= 12 \cdot (1 - g) \\
        6V - 6E + 2E &= 12 \cdot (1 - g) \\
        6V - 4E &= 12 \cdot (1 - g) \\
        4E &= 6V - 12 \cdot (1 - g) \\
        E &= 1.5V - 3 \cdot (1 - g) \\
        E &= 1.5V - c \\
    \end{aligned}
\]
Da jede Kante in zwei Vertices endet, gibt es ca. 3 Kanten pro Vertex. \\

\subsection*{(b)}

\subsection*{(c)}
Anzahl der Vertices: 14 \\
Anzahl der Faces: 17 \\

Annahme: Vertices sind zwei dimensional $\Rightarrow$ Zwei Gleitkommazahlen pro Vertex $\Rightarrow$ 16 Bytes pro Vertex \\
\[
    \begin{aligned}
        \text{Anzahl der Faces} \cdot \text{Anzahl der Vertices pro Face} \cdot \text{Bytes pro Vertex} &= \text{Bytes pro Mesh} \\    
        17 \cdot 3 \cdot 16 &= 816 \text{ Bytes pro Mesh} \\
    \end{aligned}
\]

\subsection*{(d)}
\[
    \begin{aligned}
        \text{Bytes pro Vertex} \cdot \text{Anzahl der Vertices} + \text{Anzahl der Faces} \cdot \text{Bytes pro Face} &= \text{Bytes pro Mesh} \\
        16 \cdot 14 + 17 \cdot 3 \cdot 4 &= 428 \text{ Bytes pro Mesh} \\
    \end{aligned}
\]

\subsection*{(e)}
\[
    \begin{aligned}
        \text{out} &= \text{Anzahl der Vertices} \cdot \text{Größe eines Indexes} \\
        &= 14 \cdot 4 = 56 \text{ Bytes} \\
        \text{next} &= \text{Anzahl der Halbkanten} \cdot \text{Größe eines Indexes} \\
        &= 3 \cdot \text{Anzahl der Faces} \cdot \text{Größe eines Indexes} \\
        &= 3 \cdot 17 \cdot 4 = 204 \text{ Bytes} \\
        \text{opposite} &= \text{Anzahl der Halbkanten} \cdot \text{Größe eines Indexes} \\
        &= 3 \cdot 17 \cdot 4 = 204 \text{ Bytes} \\
        \text{face} &= \text{Anzahl der Faces} \cdot \text{Anzahl der Vertices pro Face} \cdot \text{Größe eines Indexes} \\
        &= 17 \cdot 3 \cdot 4 = 204 \text{ Bytes} \\
        \text{to} &= \text{Anzahl der Halbkanten} \cdot \text{Größe eines Indexes} \\
        &= 3 \cdot 17 \cdot 4 = 204 \text{ Bytes} \\
        \text{halfedge} &= \text{Anzahl der Faces} \cdot \text{Größe eines Indexes} \\
        &= 17 \cdot 4 = 68 \text{ Bytes} \\
    \end{aligned}
\]

\subsection*{(f)}
Links: 3 \\ 
Mitte: 3

\end{document}