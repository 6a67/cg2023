\documentclass{article}
\usepackage{graphicx} % Required for inserting images
\usepackage{amsmath}
\usepackage{amssymb} % used for math symbols
\usepackage{mathtools}

\makeatletter
\newcommand*{\rom}[1]{\expandafter\@slowromancap\romannumeral #1@}
\makeatother

\title{Computer Grafik Blatt 4}
\date{May 2023}

\begin{document}

\maketitle

\section*{Aufgabe 1.}

\subsection*{(a)}
\[
eye = 
\begin{pmatrix}
0\\0\\0
\end{pmatrix}
x = 
\begin{pmatrix}
0\\0\\-32
\end{pmatrix}  
\overrightarrow{n} = 
\begin{pmatrix}
0\\0\\1
\end{pmatrix}
\]
\[
y_1 = 
\begin{pmatrix}
0\\8\\-24
\end{pmatrix}
c_{y1}= 
\begin{pmatrix}
0,5 \\ 0 \\ 0 
\end{pmatrix}
\]
\[
y_2 = 
\begin{pmatrix}
-16\\0\\-16
\end{pmatrix}
c_{y2}= 
\begin{pmatrix}
0 \\ 0,6 \\ 0 
\end{pmatrix}
\]
\[
C_a =
\begin{pmatrix}
0,2 & 0 & 0\\
0 & 0,2 & 0\\
0 & 0 & 0,2
\end{pmatrix}
\]
\[
C_d =
\begin{pmatrix}
0,8 & 0 & 0\\
0 & 0,8 & 0\\
0 & 0 & 0,8
\end{pmatrix}
\]
\[
C_s =
\begin{pmatrix}
0,4 & 0 & 0\\
0 & 0,4 & 0\\
0 & 0 & 0,4
\end{pmatrix}
\]
\[
\overrightarrow{v} = \frac{eye-x}{||eye-x||}= \frac{1}{32}\cdot
\begin{pmatrix}
0\\0\\32
\end{pmatrix}=
\begin{pmatrix}
0\\0\\1 
\end{pmatrix}
\]

\[
\overrightarrow{l_1}=\frac{y_1-x}{||y_1-x||}=\frac{1}{8\sqrt{2}}\cdot
\begin{pmatrix}
0\\8\\8
\end{pmatrix}=
\begin{pmatrix}
0\\ \frac{\sqrt{2}}{2}\\ \frac{\sqrt{2}}{2}
\end{pmatrix}
\]

\[
\overrightarrow{r_1}=2\overrightarrow{n}\overrightarrow{n}^T\overrightarrow{l_1}-\overrightarrow{l_1}=
2\overrightarrow{n}\cdot
\begin{pmatrix}
0&0&1
\end{pmatrix}
\begin{pmatrix}
0\\ \frac{\sqrt{2}}{2}\\ \frac{\sqrt{2}}{2}
\end{pmatrix}
-\overrightarrow{l_1}=\sqrt{2}\cdot \overrightarrow{n}-\overrightarrow{l_1}=
\begin{pmatrix}
0\\0 \\ \sqrt{2}
\end{pmatrix}-
\begin{pmatrix}
0\\ \frac{\sqrt{2}}{2}\\ \frac{\sqrt{2}}{2}
\end{pmatrix}=
\begin{pmatrix}
0\\ -\frac{\sqrt{2}}{2}\\ \frac{\sqrt{2}}{2}
\end{pmatrix}
\]

\[
L(x,\overrightarrow{v})=L_{y1} + L_{y2}
\]

\[
L_{y1} = (C_a + C_d\cdot(\overrightarrow{l_1}^T\overrightarrow{n})+C_s\cdot(\overrightarrow{v}^T\overrightarrow{r_1})^s)\cdot c_{y1}
\]
\[
C_{d1}=C_d\cdot(\overrightarrow{l_1}^T\overrightarrow{n})=
C_d
(\begin{pmatrix}
0& \frac{\sqrt{2}}{2}& \frac{\sqrt{2}}{2}
\end{pmatrix}
\begin{pmatrix}
0\\0\\1
\end{pmatrix})=
\begin{pmatrix}
0,8 & 0 & 0\\
0 & 0,8 & 0\\
0 & 0 & 0,8
\end{pmatrix}\cdot \frac{\sqrt{2}}{2}=
\begin{pmatrix}
\frac{2\sqrt{2}}{5} & 0 & 0\\
0 & \frac{2\sqrt{2}}{5} & 0\\
0 & 0 & \frac{2\sqrt{2}}{5}
\end{pmatrix}
\]
%% C_s1
\[
C_{s1}=C_s\cdot(\overrightarrow{v}^T\overrightarrow{r_1})^s= C_s\cdot
\begin{pmatrix}
0&0&1
\end{pmatrix}
\begin{pmatrix}
0\\ -\frac{\sqrt{2}}{2}\\ \frac{\sqrt{2}}{2}
\end{pmatrix}=
\begin{pmatrix}
0,4 & 0 & 0\\
0 & 0,4 & 0\\
0 & 0 & 0,4
\end{pmatrix}\cdot \frac{\sqrt{2}}{2}^{10}=
\begin{pmatrix}
\frac{1}{80} & 0 & 0\\
0 & \frac{1}{80} & 0\\
0 & 0 & \frac{1}{80}
\end{pmatrix}
\]
%% L_1
\[
L_1=(
C_a+C_{d1}+C_{s1})\cdot
\begin{pmatrix}
0,5\\0\\0
\end{pmatrix}=
\begin{pmatrix}
\frac{17+32\sqrt{2}}{80} & 0 & 0\\
0 &\frac{17+32\sqrt{2}}{80} & 0\\
0 & 0 & \frac{17+32\sqrt{2}}{80}
\end{pmatrix}\cdot
\begin{pmatrix}
0,5\\0\\0
\end{pmatrix}=
\begin{pmatrix}
0,3890927125\\0\\0
\end{pmatrix}
\]
%% l_2
\[
\overrightarrow{l_2}=\frac{y_2-x}{||y_2-x||}=\frac{1}{16\sqrt{2}}\cdot
\begin{pmatrix}
-16\\0\\16
\end{pmatrix}=
\begin{pmatrix}
-\frac{\sqrt{2}}{2}\\0\\ \frac{\sqrt{2}}{2}
\end{pmatrix}
\]
\[
\overrightarrow{r_2}=2\overrightarrow{n}\overrightarrow{n}^T\overrightarrow{l_2}-\overrightarrow{l_2}=
2\overrightarrow{n}\cdot
\begin{pmatrix}
0&0&1
\end{pmatrix}
\begin{pmatrix}
-\frac{\sqrt{2}}{2}\\0\\ \frac{\sqrt{2}}{2}
\end{pmatrix}-\overrightarrow{l_2}=\sqrt{2}\cdot \overrightarrow{n}-\overrightarrow{l_2} =
\begin{pmatrix}
0\\0 \\ \sqrt{2}
\end{pmatrix}-
\begin{pmatrix}
- \frac{\sqrt{2}}{2}\\0\\ \frac{\sqrt{2}}{2}
\end{pmatrix}=
\begin{pmatrix}
 \frac{\sqrt{2}}{2}\\ 0\\ \frac{\sqrt{2}}{2}
\end{pmatrix}
\]
\[
L_{y2} = (C_a + C_d\cdot(\overrightarrow{l_2}^T\overrightarrow{n})+C_s\cdot(\overrightarrow{v}^T\overrightarrow{r_2})^s)\cdot c_{y2}
\]
\[
C_{d2}=C_d\cdot(\overrightarrow{l_2}^T\overrightarrow{n})=
C_d
(\begin{pmatrix}
-\frac{\sqrt{2}}{2}&0& \frac{\sqrt{2}}{2}
\end{pmatrix}
\begin{pmatrix}
0\\0\\1
\end{pmatrix})=
\begin{pmatrix}
0,8 & 0 & 0\\
0 & 0,8 & 0\\
0 & 0 & 0,8
\end{pmatrix}\cdot \frac{\sqrt{2}}{2}=
\begin{pmatrix}
\frac{2\sqrt{2}}{5} & 0 & 0\\
0 & \frac{2\sqrt{2}}{5} & 0\\
0 & 0 & \frac{2\sqrt{2}}{5}
\end{pmatrix}
\]\[
C_{s2}=C_s\cdot(\overrightarrow{v}^T\overrightarrow{r_2})^s= C_s\cdot
\begin{pmatrix}
0&0&1
\end{pmatrix}
\begin{pmatrix}
\frac{\sqrt{2}}{2}\\ 0\\ \frac{\sqrt{2}}{2}
\end{pmatrix}=
\begin{pmatrix}
0,4 & 0 & 0\\
0 & 0,4 & 0\\
0 & 0 & 0,4
\end{pmatrix}\cdot \frac{\sqrt{2}}{2}^{10}=
\begin{pmatrix}
\frac{1}{80} & 0 & 0\\
0 & \frac{1}{80} & 0\\
0 & 0 & \frac{1}{80}
\end{pmatrix}
\]
\[
L_2=(
C_a+C_{d2}+C_{s2})\cdot
\begin{pmatrix}
0\\0,6\\0
\end{pmatrix}=
\begin{pmatrix}
\frac{17+32\sqrt{2}}{80} & 0 & 0\\
0 &\frac{17+32\sqrt{2}}{80} & 0\\
0 & 0 & \frac{17+32\sqrt{2}}{80}
\end{pmatrix}\cdot
\begin{pmatrix}
0\\0,6\\0
\end{pmatrix}=
\begin{pmatrix}
0\\0,466911255\\0
\end{pmatrix}
\]
\[
L=
\begin{pmatrix}
0,3890927125\\0\\0
\end{pmatrix}+
\begin{pmatrix}
0\\0,466911255\\0
\end{pmatrix}=
\begin{pmatrix}
0,3890927125\\0,466911255\\0
\end{pmatrix}
\]

\subsection*{(b)}
\[
\overrightarrow{h_1}=\frac{\overrightarrow{v}+\overrightarrow{l_1}}{||\overrightarrow{v}+\overrightarrow{l_1}||}=
\begin{pmatrix}
0\\ \frac{\sqrt{2}}{2}\\ \frac{2+\sqrt{2}}{2}
\end{pmatrix} \div \sqrt{\frac{4+2\sqrt2}{2}}\cdot=
\begin{pmatrix}
0\\ 0,3826834324\\0,9238795325
\end{pmatrix}
\]
\[
L_{y1} = (C_a + C_d\cdot(\overrightarrow{l_1}^T\overrightarrow{n})+C_s\cdot(\overrightarrow{h1}^T\overrightarrow{n})^s)\cdot c_{y1}
\]
\[
C_{s1}=C_s\cdot(\overrightarrow{h1}^T\overrightarrow{n})^s= C_s\cdot
\begin{pmatrix}
0& 0,3826834324&0,9238795325
\end{pmatrix}
\begin{pmatrix}
0\\ 0\\ 1
\end{pmatrix}=
\]
\[
\begin{pmatrix}
0,4 & 0 & 0\\
0 & 0,4 & 0\\
0 & 0 & 0,4
\end{pmatrix}\cdot (\frac{2+\sqrt{2}}{2} \div \sqrt{\frac{4+2\sqrt2}{2}})^{20}=
\begin{pmatrix}
0,08210449037 & 0 & 0\\
0 & 0,08210449037 & 0\\
0 & 0 & 0,08210449037
\end{pmatrix}\]
\[
L_1=(
C_a+C_{d1}+C_{s1})\cdot
\begin{pmatrix}
0,5\\0\\0
\end{pmatrix}=
\begin{pmatrix}
0,8477899153 & 0 & 0\\
0 & 0,8477899153 & 0\\
0 & 0 & 0,8477899153
\end{pmatrix}\cdot
\begin{pmatrix}
0,5\\0\\0
\end{pmatrix}=
\begin{pmatrix}
0.42389496\\0\\0
\end{pmatrix}
\]
\[
\overrightarrow{h_2}=\frac{\overrightarrow{v}+\overrightarrow{l_2}}{||\overrightarrow{v}+\overrightarrow{l_2}||}=
\begin{pmatrix}
-\frac{\sqrt{2}}{2}\\0\\ \frac{2+\sqrt{2}}{2}
\end{pmatrix} \div \sqrt{\frac{4+2\sqrt2}{2}}\cdot=
\begin{pmatrix}
-0,3826834324\\ 0\\0,9238795325
\end{pmatrix}
\]\[
C_{s2}=C_s\cdot(\overrightarrow{h2}^T\overrightarrow{n})^s= C_s\cdot
\begin{pmatrix}
-0,3826834324 & 0 &0,9238795325
\end{pmatrix}
\begin{pmatrix}
0\\ 0\\ 1
\end{pmatrix}=
\]
\[
\begin{pmatrix}
0,4 & 0 & 0\\
0 & 0,4 & 0\\
0 & 0 & 0,4
\end{pmatrix}\cdot (\frac{2+\sqrt{2}}{2} \div \sqrt{\frac{4+2\sqrt2}{2}})^{20}=
\begin{pmatrix}
0,08210449037 & 0 & 0\\
0 & 0,08210449037 & 0\\
0 & 0 & 0,08210449037
\end{pmatrix}\]
\[
L_2=(
C_a+C_{d2}+C_{s2})\cdot
\begin{pmatrix}
0\\0,6\\0
\end{pmatrix}=
\begin{pmatrix}
0.84778992 & 0 & 0\\
0 & 0.84778992 & 0\\
0 & 0 & 0.84778992
\end{pmatrix}\cdot
\begin{pmatrix}
0\\0,6\\0
\end{pmatrix}=
\begin{pmatrix}
0\\7,849262694\\0
\end{pmatrix}
\]
\[
L=
\begin{pmatrix}
0.42389496\\0\\0
\end{pmatrix}+
\begin{pmatrix}
0\\0.50867395\\0
\end{pmatrix}=
\begin{pmatrix}
0.42389496\\0.50867395\\0
\end{pmatrix}
\]

\subsection*{(c)}
\[
a=\begin{pmatrix}
1\\5\\-7
\end{pmatrix}
b=\begin{pmatrix}
-4\\4\\-7
\end{pmatrix}
c=\begin{pmatrix}
0\\0\\-7
\end{pmatrix}
p=\begin{pmatrix}
0\\3\\-7
\end{pmatrix}
\]
Da alle Punkte den gleichen Z-Wert haben, berechnen wir die Flächeninhalte in 2D.
\[
A=\frac{1}{2}det[(b-a),(c-a)]=\frac{1}{2}((-4-1)(0-5)-(0-1)(4-5))=12
\]
\[
A_{\alpha}=\frac{1}{2}det[(b-p),(c-p)]=\frac{1}{2}((-4-0)(0-3)-(0-0)(4-3))=6
\]
\[
A_{\beta}=\frac{1}{2}det[(c-p),(a-p)]=\frac{1}{2}((0-0)(5-3)-(1-0)(0-3))=1,5
\]
\[
A_{\gamma}=\frac{1}{2}det[(a-p),(b-p)]=\frac{1}{2}((1-0)(4-3)-(-4-0)(5-3))=4,5
\]
\[
\alpha = \frac{A_{\alpha}}{A}= 0,5 \;\; \beta = \frac{A_{\beta}}{A}= 0,125 \;\; \gamma = \frac{A_{\gamma}}{A}= 0,375
\]
\[
\alpha\overrightarrow{n_{\alpha}}=
\begin{pmatrix}
0\\0\\0,5
\end{pmatrix}
\beta\overrightarrow{n_{\beta}}=
\begin{pmatrix}
-0,125\\0\\0
\end{pmatrix}
\gamma\overrightarrow{n_{\gamma}}=
\begin{pmatrix}
0\\-0,375\\0
\end{pmatrix}
\]
\[
\frac{\alpha\overrightarrow{n_{\alpha}}+\beta\overrightarrow{n_{\beta}}+\gamma\overrightarrow{n_{\gamma}}}{||\cdot||}=
\begin{pmatrix}
-0.19611614 \\ -0.58834841 \\ 0.78446454
\end{pmatrix}\]
Hier sind wir uns nicht ganz sicher, ob wir den Foliensatz/die Aufgabe richtig Verstehen. Aber 
\[
phong(
\begin{pmatrix}
0\\3\\-7
\end{pmatrix},\begin{pmatrix}
-0.19611614 \\ -0.58834841 \\ 0.78446454
\end{pmatrix})=
\begin{pmatrix}
0.10096444\\ 0.12290578\\ 0
\end{pmatrix}
\]
Nach unserem Verständnis muss die Phong Beleuchtung nur für p und die aufsummierten und normierten Normalen von a, b und c ausgewertet werden.

\end{document}